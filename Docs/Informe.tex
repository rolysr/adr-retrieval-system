\documentclass[10pt]{amsart}

\usepackage[utf8]{inputenc}

\usepackage[spanish]{babel}
\usepackage{blindtext}

\usepackage{amsmath}
\usepackage{amssymb}
\usepackage{amsfonts}
\usepackage{color}
\usepackage{hyperref}
\usepackage{url}
\usepackage{stmaryrd}
\usepackage{calrsfs}
\usepackage{fancyhdr}
\usepackage{textcomp}
\usepackage{graphicx}
\usepackage{stmaryrd}
\usepackage{lipsum}

\voffset=-1.4mm
\oddsidemargin=14pt
\evensidemargin=14pt
\topmargin=26pt
\headheight=9pt     
\textheight=576pt
\textwidth=441pt %441
\parskip=0pt plus 4pt

\pagestyle{headings}
\title{}
\author{}
\date{\today}

\begin{document}
	\begin{titlepage}
		\clearpage
		\maketitle
		
		\vspace{3em}
		\begin{center}
			Proyecto: \textbf{ADR} 	
		
		\vspace{12em}
		Ejecutores: \\
		Rolando Sánchez Ramos C-311\\
		David Manuel García C-311\\
		Andry Rosquet Rodríguez C-311
		\end{center}
		\thispagestyle{empty}
	    \vspace{0.7em}
	\end{titlepage}
	\section{Introducción} \textcolor{white}{.}
	Con este informe se tratará de brindar una breve y superficial explicación relacionada al proyecto ADR-Retrieval-System perteneciente a la asignatura System-Recovery-Information. Se abordarán los modelos implementados y algunas otras particularidades. En una futura actualización se explicarán las ideas con mayor profundidad.
	
	\section{Desarrollo} \textcolor{white}{.}
	Los modelos implementados para nuestro sistema de recuperación de información son el Booelano, Vectorial y Vectorial Generalizado (el resumen de estos modelos se presentará en el informe definitivo). La implementación de estos se encuentra en el directorio llamado retrieval\_models, cada uno de ellos se encuentra en una carpeta con su propio nombre. Cada modelo se encuentra dividido en tres archivos cuyos formatos son (NOMBRE será el específico de cada modelo):
	\begin{itemize}
		\item NOMBRE\_document: define una clase NOMBREDocument la cual se encarga de crear el vector del documento deseado con un vocabulario ya definido. Para cada modelo el vector se forma de manera específica, ejemplo: el booleano crea un vector binario pero los vectoriales forman uno basado en la frecuencia de cada término. 
		\item NOMBRE\_query: define una clase NOMBREQuery que mediantes sus métodos y propiedades específicas le otorga un formato a la consulta para que sea posible aplicarle la función de similitud con un documento.  
		\item NOMBRE\_model: define una clase NOMBREModel la cual mediante sus métodos internos permite calcular la función de similitud entre un documento y la consulta. Recordar que en función del modelo, la función de similitud y operaciones necesarias serán o no diferentes.  
	\end{itemize}
	Abordemos otros elementos importantes del proyecto, estos los encontramos en el directorio utils. En este se llevaron a cabo implementaciones como: 
	\begin{itemize}
		\item El algoritmo de Rocchio que sirve para aplicarle retroalimentación a los modelos vectoriales y otorgarle importancia a la información brindada por unos documentos u otros. Este se encuentra impementado en vector\_feedback.py donde además de el método rocchio\_algorithm, se implementó retroalimentación básica para modelos vectoriales en el método classical\_vector\_feedback.
		\item El archivo query\_expansion como dice su nombre contiene la implementación de expansión de consultas basadas en sinónimos, utilizada mayormente para el modelo booleano.
		\item También se calculan las métricas basadas en la opinión de un experto, entre estas presición, recobrado, medida-f, r-presición y otras. Se pueden encontrar en metrics.py.
	\end{itemize}
	
	
\end{document}